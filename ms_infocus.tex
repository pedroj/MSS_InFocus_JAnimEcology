%%%%%%%%%%%%%%%%%%%%%%%%%%%%%%%%%%%%%%%%%%%%%%%%%%%%%%%%%%%%%%%%%%%%%%%%%%%%%%%%%%
%% MS: In Focus comment paper for Journal of Animal Ecology.
%% Draft
%% April 2016
%% Revised version: 
%% Original version wordcount: 910 words.
%%%%%%%%%%%%%%%%%%%%%%%%%%%%%%%%%%%%%%%%%%%%%%%%%%%%%%%%%%%%%%%%%%%%%%%%%%%%%%%%%%
\documentclass[a4paper,12pt]{article}
\usepackage{comment}
\usepackage{jae}
\title{IN FOCUS \\ Natural history matters: how biological constraints shape diversified interactions in pollination networks}
\running{Forbidden interactions}

\author{Pedro Jordano$^{1}$}

\affiliations{
\item Integrative Ecology Group, Estaci\'on Biol\'ogica de Do\~nana, Consejo Superior de Investigaciones Cient\'ificas (EBD-CSIC), Avenida Americo Vespucio s\slash n, E--41092 Sevilla, Spain
}

\nwords{910}
\ntables{0}
\nfig{1}
\nref{7}

\corr{\url{jordano@ebd.csic.es}}
%---------------------------------------------------------------------------------
\begin{document}

\maketitle

%\begin{abstract}
  \noindent 
  IN FOCUS: Sazatornil, F.D., Moré, M., Benitez-Vieyra, S., Cocucci, A.A., Kitching, I.J., Schlumpberger, B.O., Oliveira, P.E., Sazima, M. \& Amorim, F.W. (2016) Beyond neutral and forbidden links: morphological matches and the assembly of mutualistic hawkmoth-plant networks. Journal of Animal Ecology, 00, 000–000. doi:10.1111\/1365-2656.12509 \\
  
  \textbf{Species-specific traits and life-history characteristics constrain the ways organisms interact in nature. For example, gape-limited predators are constrained in the sizes of prey they can handle and efficiently consume. When we consider the ubiquity of such constrains it is evident how hard it can be to be a generalist partner in ecological interactions: a free living animal or plant can't simply interact with every available partner it encounters. Some pairwise interactions among coexisting species simply do not occur; they are impossible to observe despite the fact that partners coexist in the same place. Sazatornil \textit{et al.} \citep{Sazatornil:2016} explore the nature of such constraints in the mutualisms among hawkmoths and the plants they pollinate. In this iconic interacion, used by Darwin and Wallace to vividly illustrate the power of natural selection in shaping evolutionary change, both pollinators and plants are sharply constrained in their interaction modes and outcomes. } \\
%\end{abstract}

\noindent \textbf{Keywords:} complex networks, forbidden links, long-tubed flowers, mutualism, pollination, Sphingidae

\newpage

%---------------------------------------------------------------------------------
Size-limited foragers show clear restrictions on the size of prey items they can efficiently handle. In the case of plant-pollinator interactions, size uncoupling between pollinator bodies and flower sizes or structure are specially relevant in filtering out a range of potential partners \citep{Cocucci:2009}. The idea, when applied to the bizarre flowers of some plants pollinated by sphingid moths (Lepidoptera: Sphingidae), was seminal in Darwinian evolutionary theory to support the potential of natural selection in shaping adaptations \citep{Arditti:2012}. Wallace \citep{Wallace:1867} in his book, \textit{Creation by law}, vividly uses the famous example of the Malagasy orchid and its sphingid pollinator to refute the arguments of the Duke of Argyll against natural selection and Darwinism:

\begin{spacing}{1.0}
	\begin{quotation}
	 "There is a Madagascar Orchis--the \textit{Angræcum sesquipedale}--with an immensely long and deep nectary. How did such an extraordinary organ come to be developed? Mr. Darwin's [[p. 475]] explanation is this. The pollen of this flower can only be removed by the proboscis of some very large moths trying to get at the nectar at the bottom of the vessel. The moths with the longest proboscis would do this most effectually; they would be rewarded for their long noses by getting the most nectar; whilst on the other hand, the flowers with the deepest nectaries would be the best fertilized by the largest moths preferring them. Consequently, the deepest nectaried Orchids and the longest nosed moths would each confer on the other a great advantage in the 'battle of life.' This would tend to their respective perpetuation and to the constant lengthening of nectar and noses."
	 \end{quotation}
 \end{spacing}
 
 Phenotypic fitting of corolla length and shape and the pollinators' feeding apparatus and body sizes are important because the better the fit, the better the consequences in terms of fitness outcomes for the interaction partners \citep{Nilsson:1988}. Yet the expectation of perfect trait matching across populations or communities is too simplistic \citep{Anderson:2010}: "arms races" as initially suggested by Darwin and Wallace are frequently asymmetric, originating pollinator shifts rather than tight phenotypic trait matching (Fig. 1). Extensive local variation in phenotypic mismatch exists in different plant-pollinator systems \citep[e.g., ][]{Cocucci:2009,Anderson:2010,More:2012}, with pollinator-mediated selection geographic mosaics of locally coevolved partners.
 
Recent work by Sazatornil \textit{et al.} \citep{Sazatornil:2016} provides evidences that the types of trait mismatching outlined in Fig. 1 limit the ranges of host plants for sphingid pollinators, and ultimately shape their complex plant-pollinator networks. By using a comparative analysis of five different hawkmoth/flower assemblages across four South American biotas (Atlantic rainforest and Cerrado in Brazil, Chaco, and the Chaco-Yungas transition in Argentina) they tested the contributions of phenotypic matching to explain observed patterns of moth-flower interactions. 

Yet Sazatornil \textit{et al.} do not include unobserved interactions in their quantifications of mismatches, so the test of the mismatch hypothesis actually includes also forbidden links effects: a full mismatch of corolla tube/proboscis lengths actually meaning a forbidden link.

%---------------------------------------------------------------------------------

\section*{Acknowledgments}

My work was funded by a Severo-Ochoa Excellence Grant (SEV2012-0262) from the Spanish Ministerio de Econom\'ia y Competitividad (MINECO), and RNM-5731 from the Junta de Andaluc\'ia. Andrea Cocucci generously provided material for Fig. 1 and insightful discussions on sphingids and long-tubed flowers.

\newpage

%---------------------------------------------------------------------------------
% Bibliography
% \bibliography{ms_infocus}
%%% Unnumbered Literature Cited

\begin{thebibliography}{10}

\bibitem{Sazatornil:2016}
Sazatornil, F.D., Moré, M., Benitez-Vieyra, S., Cocucci, A.A., Kitching, I.J., Schlumpberger, B.O., Oliveira, P.E., Sazima, M. \& Amorim, F.W. (2016) Beyond neutral and forbidden links: morphological matches and the assembly of mutualistic hawkmoth-plant networks. Journal of Animal Ecology, 00, 000–000.

\bibitem{Cocucci:2009}
Cocucci, A.A., Mor\'e, M. \& S\'ersic, A.N. (2009) Restricciones mec\'anicas en las interacciones planta-polinizador: estudio de casos en plantas polinizadas por esf\'ingidos. Interacciones planta---animal y la conservaci\'on de la biodiversidad (eds R. Medel, R. Zamora, M. Aizen \& R. Dirzo), pp. 43–59. CYTED, Madrid.

\bibitem{Arditti:2012}
Arditti, J., Elliott, J., Kitching, I.J. \& Wasserthal, L.T. (2012) “Good Heavens what insect can suck it”–Charles Darwin, \textit{Angraecum sesquipedale} and \textit{Xanthopan morganii praedicta}. Botanical Journal of the Linnean Society, 169, 403–432.

\bibitem{Wallace:1867}
Wallace, A.R. (1867) Creation by law. The Quarterly Journal of Science, 4, 471–488.

\bibitem{Nilsson:1988}
Nilsson, L.A. (1988) The evolution of flowers with deep corolla tubes. Nature, 334, 147–149.

\bibitem{Anderson:2010}
Anderson, B., Terblanche, J.S. \& Ellis, A.G. (2010) Predictable patterns of trait mismatches between interacting plants and insects. BMC Evolutionary Biology, 10, 204.

\bibitem{More:2012}
Moré, M., Amorim, F.W., Benitez-Vieyra, S., Medina, A.M., Sazima, M. \& Cocucci, A.A. (2012) Armament imbalances: match and mismatch in plant-pollinator traits of highly specialized long-spurred orchids (ed M. Heil). PLoS ONE, 7, e41878.

\bibitem{BasJor:2014}
Bascompte, J. \& Jordano, P. (2014) Mutualistic Networks. Princeton University Press, Princeton, NJ.

\end{thebibliography}

%---------------------------------------------------------------------------------

\begin{flushright}
  \noindent 
  		\textbf{Pedro Jordano }\\
  		\begin{spacing}{1.0}
		\textit{Integrative Ecology Group, Estaci\'on Biol\'ogica de Do\~nana, \\ Consejo Superior de Investigaciones Cient\'ificas (EBD-CSIC), \\ Avenida Americo Vespucio s\slash n, \\ E--41092 Sevilla, Spain}
		\end{spacing}
\end{flushright}
\newpage

%---------------------------------------------------------------------------------
\section*{Figures}


\begin{figure}[h!]
  \caption{Morphological mismatches set important biological constraints for size-limited foragers, including e.g., predators, pollinators, and frugivores. In plant-animal mutualisms, a morphological mismatch between partners sets size limits that filter out a range of phenotypes that otherwise could eventually interact. Other reasons for forbidden links include, e.g., phenological differences \citep{BasJor:2014}. Thus, a number of the potential interactions that could take place in a given mutualistic assemblage simply cannot occur because of biological reasons: these are forbidden interactions. Photo: Andrea Cocucci. An sphingid moth, \textit{Agrius cingulata}, visiting a flower of \textit{Bauhinia mollis} (Fabaceae), Las Yungas, Argentina.}
  \label{Fig1}
  \begin{center}
    \includegraphics[width=14cm]{Fig1}
  \end{center}
\end{figure}
%---------------------------------------------------------------------------------
%%%%%%%%%%%%%%%%%%%%%%%%%%%%%%%%%%%%%%%%%%%%%%%%%%%%%%%%%%%%%%%%%%%%%%%%% MS NOTES
\begin{comment}
\bibitem{Borrell:2005}
Borrell, B. (2005) Long tongues and loose niches: Evolution of euglossine bees and their nectar flowers. Biotropica, 37, 664–669.

\bibitem{More:2012}
Mor\'e, M., Amorim, F.W., Benitez-Vieyra, S., Medina, A.M., Sazima, M. \& Cocucci, A.A. (2012) Armament imbalances: match and mismatch in plant-pollinator traits of highly specialized long-spurred orchids. PLoS ONE, 7, e41878. 


\end{comment}
%%%%%%%%%%%%%%%%%%%%%%%%%%%%%%%%%%%%%%%%%%%%%%%%%%%%%%%%%%%%%%%%%%%%%%%%%%%%%%%%%%
\end{document}

